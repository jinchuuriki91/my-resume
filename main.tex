\documentclass[10pt,a4paper]{altacv}

\geometry{left=1.25cm,right=1.25cm,top=1.5cm,bottom=1.5cm,columnsep=1.2cm}

\usepackage[default]{lato}
\usepackage[T1]{fontenc}
\usepackage[utf8]{inputenc}

\definecolor{DarkGrey}{HTML}{111111}
\definecolor{SlateGrey}{HTML}{2E2E2E}
\definecolor{LightGrey}{HTML}{666666}

\colorlet{name}{DarkGrey}
\colorlet{tagline}{SlateGrey}
\colorlet{heading}{DarkGrey}
\colorlet{headingrule}{DarkGrey}
\colorlet{accent}{DarkGrey}
\colorlet{emphasis}{SlateGrey}
\colorlet{body}{LightGrey}

\begin{document}

\name{Gandhar Pednekar}
\tagline{Pythonista}
\photoR{3.5cm}{image}
\personalinfo{
  \email{gandhar.pednekar15@gmail.com}
  \phone{(+91) 954-594-4131}
  \homepage{http://gandharp.xyz}
  \homepage{https://leetcode.com/u/gandhar169/}
  \linkedin{https://www.linkedin.com/in/pednekargandhar/}
  \github{https://github.com/jinchuuriki91}
  \location{Bengaluru, IN}
}

\makecvheader

\cvsection{Skills}
\printinfo{Programming Languages:}\\
\cvtag{Python}
\cvtag{Java 8}
\cvtag{Javascript}
\cvtag{Typescript}
\cvtag{SQL}

\divider\smallskip

\printinfo{Web Development:}\\
\cvtag{React}
\cvtag{Next.js}
\cvtag{Angular}
\cvtag{Material UI}
\cvtag{HTML}
\cvtag{CSS}

\divider\smallskip

\printinfo{Frameworks:}\\
\cvtag{Django}
\cvtag{FastAPI}
\cvtag{Flask}
\cvtag{Express}
\cvtag{Dropwizard}
\cvtag{Docker}

\divider\smallskip

\printinfo{Cloud computing platforms:}\\
\cvtag{AWS}
\cvtag{Firebase}
\cvtag{DigitalOcean}
\cvtag{Heroku}

\cvsection{Experience}

\cvevent{Senior Full Stack Engineer}{XTRA Vision AI}{Feb 2021 -- Present}{Remote - Bengaluru, IN}
\begin{itemize}
\item As Founding team member, got the opportunity to setup the product from scratch.
\item Built the PaaS and SaaS webapps, API services and database schemas.
\item Architected end to end AWS infrastructure for scalability across 3 major region, N. Virginia, London and Mumbai.
\item Used AWS ECS to setup production, staging and testing environments.
\item Since this was the first time I worked on such complex infrastructure, learnt a lot about security on the cloud.
\item Using Bitbucket, setup CI/CD pipelines. All microservices were Dockerized.
\item Contributed to XTRA's developer SDKs in Javascript, Flutter, Kotlin.
\item Contributed to internal API documentation.
\end{itemize}

\divider\smallskip

\cvevent{Member of Technical Staff 2}{Fyle}{Sep 2019 -- Sep 2020}{Bengaluru, IN}
\begin{itemize}
\item Worked on implementing Customer requested features into the platform
\item Got exposure to AngularJS for WebApp and Dropwizard a Java framework for building REST APIs
\item Hands-on experience with building microservices using Docker
\item Wrote performant SQL queries for retrieving API response from resources
\item Wrote Design Docs for the feature being implemented, this was a good learning experience
\item Built a slack bot for Fyle platform so users can interact from Slack itself
\item Worked on Gmail Add on built using Google App Script which helped users to add expenses from Gmail
\end{itemize}

\newpage

\cvevent{Software Engineer}{Reckonsys Tech Labs}{Dec 2018 -- Aug 2019}{Bengaluru, IN}
\begin{itemize}
\item Assigned to a project building a Logistics management SaaS platform
\item Primary role was to plan and execute the implementation
\item This included communicating with the client wrt project requirements, stages of deliverables and scale of usage
\item Implemented using Angular, Django and PostgreSQL
\item Led a team of 4 engineers
\item Architected database schema and infra setup for production and testing
\item Worked with Django to build REST APIs and Angular for WebApp
\end{itemize}

\divider\smallskip

\cvevent{Full Stack Developer}{Finesse Digital}{Oct 2018 -- Nov 2018}{Remote}
\begin{itemize}
\item First exposure to working as a remote developer for a Singapore based startup
\item Mostly worked on backend REST APIs in Django framework
\item Included writing automated test cases through pipelines
\end{itemize}

\divider\smallskip

\cvevent{Software Engineer}{InTouchApp}{Jan 2016 -- Oct 2018}{Pune, IN}
\begin{itemize}
\item Started off as a Front End developer for first 6 months but due to my interest in the
Python started working with Backend Team resulting in a Full Stack experience
\item As a Front End developer I’ve worked on InTouchApp WebApp and HTML email templates
\item As a Backend developer I’ve worked extensively in Django 1.9, tasks included
creating APIs, writing tests, database schema architecture and monitoring
\item Built an analytics internal tool using a time series database called InfluxDB
\item Integrated client analytics SDK tool Countly with InfluxDB, no library out does this
\item Implemented an incremental backup script for our InfluxDB arch since
Open source version of above is void of it
\item Written Chef recipes AWS EC2 Setup and Configure
\item Setup Bitbucket pipelines for automated testing upon merge of a feature branch
\item Web crawlers for mining data for InTouchApp features … This was fun! :)
\end{itemize}

\divider\smallskip

\cvevent{Front End Engineer}{Leova}{Jul 2015 -- Dec 2015}{Hyderabad, IN}
\begin{itemize}
\item Built web interface and custom css grid for Leova Flight API developer console
\item Javascript wrapper for searching flights on Kayak, Expedia and Hipmunk
\item Demo Chromecast app using Leova Flight API
\item Leova forum for platform users to post Q\&A regarding Leova Flight API
\item Built analytics tool for Leova Flight API
\item Web scrapper to extract food items data from eat24.com
\item Python script to organise the scrapped data for Leova Food API
\item Virtual File System for faster web development on Google App Engine
\end{itemize}

\newpage

\cvevent{Intern}{WeChat}{Jul 2014 -- Sept 2014}{Pune, IN}

\smallskip

\cvsection{Education}

\cvevent{B.S.\ in Data Science and Applications}{Indian Institute of Technology, Madras}{May, 2024 -- June, 2028}{}

\divider

\cvevent{B.E.\ in Computer Engineering}{Maharashtra Institute of Technology, Pune}{June, 2010 -- June, 2015}{}

\divider

\cvevent{H.S.C.}{RCF Jr. College, Alibag}{}{}

\divider

\cvevent{S.S.C.}{St. Mary's Convent School, Alibag}{}{}

\end{document}


